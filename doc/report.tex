\documentclass{article}
\usepackage{booktabs}
\usepackage{float}
\usepackage{fancyhdr}
\usepackage{amsmath}
\usepackage[nottoc]{tocbibind}
\usepackage{hyperref}
\usepackage{url}
\hypersetup{
    pdfborder = {0 0 0}
}
\fancyhf{} % Clear header/footer
\renewcommand{\headrulewidth}{0pt} % Remove header rule
\usepackage[english]{babel}

\usepackage{appendix}

\usepackage{listings}
\usepackage{xcolor}
\usepackage{color}

\definecolor{json_string}{RGB}{160,32,240}
\definecolor{json_keyword}{RGB}{0,0,255}
\definecolor{json_number}{RGB}{0,128,0}

\lstdefinelanguage{json}{
    basicstyle=\ttfamily\footnotesize,
    numbers=left,
    numberstyle=\tiny,
    stepnumber=1,
    numbersep=5pt,
    showstringspaces=false,
    breaklines=true,
    frame=lines,
    backgroundcolor=\color{white},
    keywords=[2]{true,false,null},
    keywordstyle=[2]\color{json_keyword},
    keywords=[3]{string,number,array,object},
    keywordstyle=[3]\color{json_keyword},
    string=[b]{"},
    stringstyle=\color{json_string},
    comment=[l]{//},
    morecomment=[s]{/*}{*/},
    commentstyle=\color{gray},
    tabsize=2
}



\begin{document}

\title{RAFT consensus algorithm over a distributed Key-Value store}
\author{
Kalopisis Ioannis 7115112200011\\
Notaris  Ioannis 7115112200038}
\date{\today}


\maketitle
\newpage
\tableofcontents

\newpage

\section{Introduction}
Distributed systems have gained immense popularity in recent years due to their ability to
handle large-scale data storage and processing.\ Key-value stores, a fundamental building
block of many distributed systems, provide a simple and efficient way to store and retrieve
data based on unique keys.\ In this report, we present the design and implementation of a
key-value store that utilizes the Raft consensus algorithm for fault-tolerant replication
and consistency across a cluster of nodes.

The key-value store we have developed serves as a reliable and scalable solution for storing
and retrieving key-value pairs in a distributed environment.\ The system is designed to handle
high volumes of read and write requests while ensuring strong consistency guarantees.
By leveraging the Raft consensus algorithm, our system achieves fault tolerance and replication,
enabling it to operate reliably even in the face of node failures or network partitions.

The core functionality of our system revolves around the interaction between a client program,
the key-value nodes, and the Raft algorithm.\ Clients send messages to the key-value store,
which then forwards them to the Raft consensus algorithm.\ Raft ensures that the messages are
replicated across a specified number of nodes in the cluster, maintaining consistency and
fault tolerance.\ Once a response is obtained from the key-value nodes, it is sent back to the
client program.\ This process enables clients to interact seamlessly with the key-value store,
enjoying the benefits of fault tolerance, scalability, and consistency provided by the
underlying Raft algorithm.

In the following sections of this report, we will delve deeper into the design choices and
implementation details of our key-value store and the integration of the Raft consensus algorithm.
We will explore the architecture, communication protocols, and fault-tolerance mechanisms employed
in the system.\  Ultimately, our goal is to present a comprehensive overview of our key-value
store and Raft implementation, showcasing the robustness and efficiency of our distributed system.


\section{External Client}


\section{Key-Value store}
The Key-Value Store is a distributed server-based application that facilitates the storage
and retrieval of data using unique key-value pairs.\ It provides a simple yet powerful
abstraction for organizing and accessing data efficiently.\ This section provides a comprehensive
overview of the Key-Value Store application, including its purpose, architecture, functionality,
advantages, and potential use cases.

The purpose of the Key-Value Store application is to offer a scalable and efficient solution
for data storage and retrieval.\ By using a key-value pair approach, it enables clients to
easily store and retrieve data using unique keys.


\subsection{Architecture}
The Key-Value Store application follows a distributed architecture that ensures high availability,
fault tolerance, and scalability.\ Data is distributed and replicated across multiple servers,
with each server responsible for a specific portion of the key space.\ Clients interact with the
servers to perform operations like storing, retrieving, updating, and deleting key-value
pairs.\ The servers communicate with each other, through Raft system if needed, to maintain
data consistency and handle server failures.

\subsection{Communication}
At this section we will describe the communication between the client and the key-value server,
and the communication between the key-value server and the Raft server.\ Also we will describe
the message types that are exchanged between the client, key-value server, and Raft server.

\subsubsection{Message Types}
The client, key-value server, and Raft server exchange messages in the form of JSON structures.
Two classes were implemented to better decode, manage, and store the messages exchanged between
the client, key-value server, and Raft server.\ These two classes are ServerJSON and RaftJSON.\
ServerJSON is a data structure used for communication between the client and the key-value
server.\ It typically contains fields such as the command, and sender of the message.\ When
the client sends a request to the key-value server, it encapsulates the necessary information
in a ServerJSON message and sends it over the network.

\begin{lstlisting}[language=json, caption={ServerJSON},label={lst:ServerJSON}]
{
    "sender": "CLIENT | KV_SERVER",
    "commands": "example command"
}
\end{lstlisting}

The sender field can only have two values, CLIENT or KV\_SERVER, which indicate whether the
message was sent by the client or the key-value server, respectively.\ The commands field
contains the command to be executed by the key-value server.\ The command can be one of
the following:
\begin{itemize}
    \item \text{PUT}: store a key-value pair in the key-value store.
    \item \text{DELETE}: delete a key-value pair from the key-value store.
    \item \text{SEARCH}: retrieve a key-value pair that match the specific pattern that
                           was given to the system
\end{itemize}

On the other hand, RaftJSON is used for communication between the key-value server and the
Raft server.\ It contains an extra field, which is a list of key-value server IDs, to indicate
on which servers of the key-value store the data-import commands should be executed.\ Also
the commands field is, unlike ServerJSON, an array of command strings.

\begin{lstlisting}[language=json, caption={RaftJSON},label={lst:RaftJSON}]
{
    "sender": "RAFT",
    "commands": ["example command 1",
                 "example command 2",
                 "example command 3"],
    "rep_ids": [1, 5, 6]
}
\end{lstlisting}

Both ServerJSON and RaftJSON messages are typically serialized into a suitable data format,
such as JSON, before transmission over the network.\ This allows the messages to be easily
parsed and reconstructed by the receiving party.\ The exchange of messages between the client,
key-value server, and Raft server enables the coordination, synchronization, and fault tolerance
mechanisms necessary for the reliable operation of the Key-Value Store application.

\subsubsection{Communication Schema}
The communication schema and path between the client, key-value server, and Raft server in the
Key-Value Store application can be summarized in the following steps:
\begin{itemize}
    \item \text{Client to Key-Value Server}: The client communicates with the Key-Value Server
    by sending requests over a network connection.\ When a client wants to store or retrieve
    a key-value pair, it sends a request to the Key-Value Server.\ The request is a ServerJSON
    object.\ The Key-Value Server receives the request and processes it by executing the
    appropriate action based on sender and the command type (PUT, DELETE, SEARCH).\ If the
    command type is PUT or DELETE, that is, some change should be made to the data stored on
    the servers, the Key-Value server reformat the message and send to Raft.\ Once the
    action is completed, the Key-Value Server sends a response back to the client.
    \item \text{Key-Value Server to Raft Server}: The Key-Value Server communicates with the
    Raft Server for maintaining consistency and fault tolerance.\ When a Key-Value Server
    receives a client request that requires updates to the key-value data, it interacts with
    the Raft Server to ensure that the updates are replicated across the cluster of Key-Value
    Servers.\ It sends append entries requests to the Raft Server, which includes the updates
    to the key-value data.\ The payload of the API call is a RaftServer object.\ The Raft
    Server replicates these updates to all servers and send the payload to all Key-Value
    Servers.
    \item \text{Raft Server to Key-Value Server}: The Raft Server communicates with the Key-Value
    Servers to maintain coordination and provide fault tolerance.\ The Raft Server receive the
    payload from the append entries API call and replicate it to all Raft nodes.\ It then send
    the message to all Key-Value Servers to process it.\ The payload of the API call is a string
    of RaftJSON object as said previously.
\end{itemize}

Overall, the communication between the client, key-value server, and Raft server in the Key-Value
Store application follows a distributed architecture where the client interacts directly with the
key-value server for data operations, while the key-value server interacts with the Raft server
for consistency and fault tolerance.\ The Raft server coordinates the replication of updates and
ensures that the key-value servers are in sync.\ This communication schema enables a reliable and
scalable distributed key-value storage system.


\section{RAFT consensus algorithm}
The implementation consists of two primary components: the \texttt{RaftServer} and 
the \texttt{Log} class. Together, these components implement the Raft consensus 
algorithm. The user interacts with the raft servers via an API from the cli. This report provides a detailed overview of 
how the Raft consensus algorithm is implemented in the provided code.

\begin{enumerate}
	\item \textbf{Raft Server Class:}
	The \texttt{raft\_node} class represents an individual node in the Raft consensus algorithm. It encapsulates the logic for managing the node's state, participating in leader election, replicating log entries, and handling client requests. The key features of the \texttt{raft\_node} class are as follows:
	\begin{enumerate}
		\item Configuration: The \texttt{raft\_config} object, an instance of the IniConfig class, is used to store and access configuration parameters for the Raft node. These parameters include timeouts and election-related settings. Additionally, the servers object, an instance of the JsonConfig class, holds the details of all servers participating in the Raft cluster, such as their host addresses and ports.
		\item RPC and Communication: The Raft node utilizes the RPCClient object to communicate with other nodes in the Raft cluster using remote procedure calls (RPCs). This enables the node to send and receive messages for leader election, log replication, and handling client requests.
		\item State and State Transitions: The \texttt{raft\_node} class maintains various state variables, including the current node's role (leader, follower, or candidate), term, and voted-for candidate. The node transitions between different states based on internal events and external interactions with other nodes. The implementation includes methods for handling state transitions, such as \texttt{start\_election()}, \texttt{transition\_to\_follower()}, \texttt{transition\_to\_leader()}, \texttt{transition\\\_to\_candidate()} etc.
		\item Leader Election: The Raft node participates in leader election by exchanging messages with other nodes in the cluster. It utilizes a randomized election timeout mechanism to trigger leader election if no heartbeat messages are received within the timeout duration. The \texttt{raft\_node} class includes \texttt{start\_election()} method for initiating leader election and handling vote responses and processing vote requests (\texttt{request\_vote\_rpc()}).
		\item Log Replication: The \texttt{raft\_node} class implements the log replication mechanism to ensure consistency across all nodes in the cluster. It includes methods for appending new log entries to the log, replicating log entries to followers, and committing log entries once a majority of nodes have acknowledged them. The log replication is achieved through interactions with the Log class, which manages the log entries and interacts with a MongoDB database for persistence.
		\item Append Entries Mechanism: The Raft consensus algorithm utilizes the append entries mechanism to replicate log entries across nodes in the cluster. The main \texttt{run} method is responsible for sending append entries RPCs to followers, updating their logs, and ensuring consistency. The method includes parameters such as the leader's term, leader ID, previous log index, previous log term, and a list of log entries to be replicated. Followers validate the received entries, append them to their logs, and respond with success or failure status (\texttt{append\_entries\_rpc()}).
		\item Client Requests: The Raft node handles client requests by forwarding them to the leader node for processing via the API.
	\end{enumerate}
	\item \textbf{Log Class:}
	The Log class is responsible for managing the log entries of the Raft node. It interacts with a MongoDB database to store and retrieve log entries efficiently. The key functionalities of the Log class are as follows:
	\begin{enumerate}
		\item Initialization and Database Connection: The Log class is initialized with the necessary parameters, including the database URI, database name, collection name, and server ID. It establishes a connection to the MongoDB database using the provided URI and initializes the collection for storing log entries.
		\item Loading and Saving Entries: The \texttt{load\_entries()} method retrieves all log entries from the MongoDB collection and populates the entries list of the Log instance. The \texttt{save\_entry()} method is used to store a new log entry in the MongoDB collection.
		\item Log Entry Management: The Log class provides methods to append new log entries (\texttt{append\_entry()}), retrieve a specific entry by index (\texttt{get\_entry()}), and retrieve the last index (\texttt{get\_last\_index()}) and term (\texttt{get\_last\_term()}) of the log. It also includes methods to mark an entry as committed (\texttt{commit\_entry()}), delete entries from the collection (\texttt{delete\_entry\_from\_collection()} and \texttt{delete\_entries\\\_after()}).
		\item Committing Entries: The \texttt{commit\_entries()} method is responsible for marking log entries as committed within a specified range. It iterates over the entries in the given range, updates their \texttt{is\_committed} flag, and updates the corresponding documents in the MongoDB collection. Additionally, it invokes the \texttt{append\_to\_state\_machine()} method to apply the committed entries to the state machine.
		\item Querying Log Entries: The \texttt{get\_all\_entries\_from\_index()} method returns all log entries from a specified index onwards. The \texttt{is\_empty()} method checks if the log is empty by examining the length of the entries list.
		\item State Machine Interaction: The \texttt{append\_to\_state\_machine()} method invokes the \texttt{raft\_request} RPC on the key-value store server using the \texttt{kv\_store\_rpc\_client} object. This allows the log entry to be applied to the state machine for further processing.
	\end{enumerate}
\end{enumerate}


\section{CLI}


% \newpage
% \bibliographystyle{unsrt}
% \bibliography{bib}

\end{document}



